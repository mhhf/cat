% -*- coding:utf-8 -*-
\input{header.tex}

\newcommand{\uuid}{}
\newcommand{\localtag}{}
\newcommand{\globaltag}{}
\newcommand{\tags}[1]{
    \ifnote
        \renewcommand{\localtag}{#1}
    \else
        \renewcommand{\globaltag}{#1}
    \fi
    }
    \newif\ifnote
\newenvironment{note}
{\par\medskip\noindent\hspace*{0.3in}
  \tabular{!{\color{blue!40}%
      \vrule width 5pt}p{\dimexpr\linewidth-0.6in-2\tabcolsep-5pt\relax}|}\hline}
{\\\hline\endtabular\par\medskip}
\newenvironment{field}{\center}{\center}
\newcommand{\xplain}[1]{\renewcommand{\uuid}{#1}}
\begin{document}

\title{Category Theory Anki Study Document}
\author{Denis Erfurt}
\maketitle
\tags{cat}
% TODO - neutral elements with their coresponding connnectives

\begin{note}
  \xplain{6fd37e54-117f-4f05-be5b-55a0c4dc3441}
  \begin{field}
    Def. Category
  \end{field} \\
  \noindent\rule[0.5ex]{\linewidth}{1pt}
  \begin{field}
    \begin{itemize}
      \item Objects: A, B, C, ...
      \item Arrows: f, g, h, ...
      \item for each $f:A\rightarrow B$ \\
        domain: A = dom(f)\\
        codomain: B = cod(f)
      \item for $f: A \rightarrow B$, $g: B \rightarrow C$ with $cod(f) = dom(g)$\\
        composite of $f$ and $g$: $g\circ f: A\rightarrow C$
      \item for each object A: \\
        identity arrow: $1_A: A\rightarrow A$
      \item Associativity:
          $h\circ(g\circ f) = (h\circ g)\circ f$
      \item Unit: $f\circ 1_A = f = 1_B\circ f$
    \end{itemize}
  \end{field}
\end{note}

\begin{note}
  \xplain{3b29dd61-7775-4d74-b1cd-d2010886386d}
  \begin{field}
    Def. concrete Categories
  \end{field} \\
  \noindent\rule[0.5ex]{\linewidth}{1pt}
  \begin{field}
    Categories in which Objects are Sets, possibly equipped with some structure, and arrows are certain, possibly structure-perserving, functions.
  \end{field}
\end{note}

\begin{note}
  \xplain{4563a002-2ed0-4dab-aa0c-351899c215da}
  \begin{field}
    Def. functor
  \end{field} \\
  \noindent\rule[0.5ex]{\linewidth}{1pt}
  \begin{field}
    Let \textbf{C}, \textbf{D} be categories, then $F: \textbf{C} \rightarrow \textbf{D}$ is a functor with:\\
    \begin{enumerate}
      \item $F(f: A \rightarrow B) = F(f): F(A) \rightarrow F(B)$
      \item $F(g\circ f) = F(g)\circ F(f)$
      \item $F(1_A) = 1_{F(A)}$
    \end{enumerate}
  \end{field}
\end{note}

\begin{note}
  \xplain{e5e5cf5e-cf15-4de9-8757-d24b5baabdc3}
  \begin{field}
    Def. discrete categories
  \end{field} \\
  \noindent\rule[0.5ex]{\linewidth}{1pt}
  \begin{field}
    categories with only the identity arrows
  \end{field}
\end{note}

\begin{note}
  \xplain{e622e5f4-656d-4a9f-a1fe-a39a0f803938}
  \begin{field}
    Def. monoid
  \end{field} \\
  \noindent\rule[0.5ex]{\linewidth}{1pt}
  \begin{field}
    A set M with an associative binary operation $\cdot: M\times M \rightarrow M$ and unit element $u\in M$.
  \end{field}
\end{note}

\begin{note}
  \xplain{057e6244-db87-4f91-a428-3c1eaef53cce}
  \begin{field}
    Def. isomorphism
  \end{field} \\
  \noindent\rule[0.5ex]{\linewidth}{1pt}
  \begin{field}
    In any category \textbf{C}, an arrof $f: A\rightarrow B$ is an isomorphism, if there is an arrow $g: B\rightarrow A$ in \textbf{C} such that.
    \[ g\circ f = 1_A\text{ and }f\circ g = 1_B \]
    we write $f^{-1} = g$.
    A is isomorphic to B ($A\cong B$) if there exists an isomorphism between them.
  \end{field}
\end{note}

\begin{note}
  \xplain{c9265ab2-0592-4d3b-a9f3-e89b0ce35a50}
  \begin{field}
    Def. group
  \end{field} \\
  \noindent\rule[0.5ex]{\linewidth}{1pt}
  \begin{field}
    A group G is a monoid with an inverse $g^{-1}$ for every element g.
  \end{field}
\end{note}

\begin{note}
  \xplain{a74acbfb-3d45-4205-b9ce-b1d2a29c0cd6}
  \begin{field}
    Def. Free Monoid
  \end{field} \\
  \noindent\rule[0.5ex]{\linewidth}{1pt}
  \begin{field}
    A monoid M is \textbf{freely generated} by a subset A of M with:
    \begin{enumerate}
      \item \textbf{no junk}: every element $m\in M$ can be written as a product of elemens of A\\
        \[ m = a_1 \cdot_M ... \cdot_M a_n,\ \ \ a_i\in M \]
      \item \textbf{no noise}: No "nontrivial" relations hold in M: if $a_1 ... a_n = a_1' ... a_n'$ then this is required by the axioms for monoids.
    \end{enumerate}
  \end{field}
\end{note}

\begin{note}
  \xplain{a04e8712-787b-4914-94e1-ab0e0461661b}
  \begin{field}
    Def. Universal Mapping Propperty M(A)
  \end{field} \\
  \noindent\rule[0.5ex]{\linewidth}{1pt}
  \begin{field}
    given $i: A \rightarrow |M(A)|$, Monoid N and $f: A \rightarrow |N|$ there is a unique monoid homomorphism $\bar f: M(A) \rightarrow N$ s.t. $|\bar f|\circ i = f$
  \end{field}
\end{note}
















\end{document}
